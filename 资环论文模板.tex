\documentclass[12pt, a4paper]{article}

% 引入必要的宏包
\usepackage[UTF8]{ctex} % 支持中文处理,并指定UTF8编码
\usepackage{graphicx} % 用于插入图片
\usepackage{geometry} % 用于设置页面边距
\usepackage{tabularx} % 用于创建更灵活的表格
\usepackage{array}    % 用于更高级的表格列格式
\usepackage{fontspec} % 用于字体设置
\usepackage{amsmath}  % 数学公式
\usepackage{fancyhdr} % 用于页眉页脚 
\usepackage{titlesec} % 用于自定义标题样式
\usepackage{setspace} % 用于设置行间距
\usepackage{tocloft} % 用于自定义目录样式
\usepackage{gbt7714} % 支持国标参考文献样式
\usepackage{xcolor} % 支持颜色设置
\usepackage{lastpage} % 用于获取总页数
\renewcommand*{\refname}{} % 抑制gbt7714的参考文献标题


% 页面边距设置
\geometry{left=2.5cm, right=2.5cm, top=2.5cm, bottom=2.5cm}

% 去掉页眉页脚和页码 (封面不需要)
\pagestyle{empty}

\begin{document}

\begin{center}
    \includegraphics[height=2.94cm, width=14.64cm, keepaspectratio]{image/名字.png} %
\end{center}

\vspace{1.5cm}

% 毕业设计标题
\begin{center}
    {\zihao{-1} \songti \textbf{本科毕业设计(论文})}
\end{center}

\vspace{2cm}

% 校徽
\begin{center}
    \includegraphics[height=4.64cm, width=8.25cm, keepaspectratio]{image/校徽.png} 
\end{center}

\vspace{1cm}

% 信息表格
% 修改命令定义,使用 \makebox 实现左列两端对齐
\newcommand{\tableitem}[2]{{\zihao{3}\songti \makebox[3.5cm][s]{#1}} & {\zihao{3}\songti #2} \\ \cline{2-2}}
\newcommand{\tableitemshort}[2]{{\zihao{3}\songti \makebox[3.5cm][s]{#1}} & {\zihao{3}\songti #2} \\ \cline{2-2}} % 题目行也应用此格式

\begin{center} % 表格居中
\noindent % 防止表格前有缩进
\setlength{\tabcolsep}{2.69pt} % 设置单元格左右边距
\renewcommand{\arraystretch}{1.5} % 增加行高倍数
% 左列通过 \makebox 实现两端对齐,右列居中
\begin{tabularx}{13.19cm}{l >{\centering\arraybackslash}X} % 左列改为l,对齐由makebox控制
    \tableitemshort{题    目:}{XXXXXXXX的发展现状及应用研究} 
    \tableitem{学    院:}{XX学院} 
    \tableitem{专    业:}{}  
    \tableitem{年 级、班:}{202X级 XXX班} 
    \tableitem{学生姓名:}{XX} 
    \tableitem{指导教师:}{XX} 
\end{tabularx}
\end{center} % 表格居中结束

\vspace{1cm}

\vspace{1cm}
% 日期
\begin{center}
    {\zihao{3} \heiti 二〇二×年×月×日}
\end{center}

\newpage % 封面结束,开始新页面
% 从摘要页开始设置页眉
\pagestyle{fancy} % 设置页眉页脚样式
\fancyhf{} % 清空当前的页眉页脚
\renewcommand{\headrulewidth}{0pt} % 去掉页眉分割线
\renewcommand{\footrulewidth}{0pt} % 去掉页脚分割线
\chead{\color{gray}\songti\zihao{5}兰州资源环境职业技术大学毕业论文} % 设置页眉居中内容,宋体小五灰色
%\cfoot{\thepage} % 页脚中间显示页码 (封面、摘要、目录前不显示页码)

% 论文标题(中文页)
\begin{center}
    {\zihao{-2}\heiti XXXXXXXX 的发展现状及应用研究}
\end{center}

\vspace{1.5cm}

% 摘要标题和内容
\noindent % 取消段落缩进
{\heiti\zihao{4}摘要}

\vspace{0.5\baselineskip} % 段前0.5行
\begin{spacing}{1.25} % 设置1.25倍行距
\songti\zihao{-4} % 将字体和字号设置移到这里
龙,是中华民族的图腾,也是中华文化的象征,它深深地烙印在每一个华夏儿女的心中。板凳龙是中国舞龙的形式之一,它的造型花样繁多,但整体形式基本相同,由龙头、龙身、龙尾和一条条长板凳连接而成,融合了体育、杂技、舞蹈等元素,具有很高的观赏价值和文化价值。舞龙者一般为20-50周岁男性,盘龙时,由龙头带领龙身和龙尾呈圆盘状进行表演,这就需要舞龙者具有良好的身体协调能力、灵活性与敏捷性、较好的体力与耐力以及舞者之间的默契配合,以便舞龙者在舞龙过程中准确地完成各种动作。
\end{spacing}

\vspace{0.5\baselineskip} % 段后0.5行

% 关键词
\noindent % 取消段落缩进
{\heiti\zihao{4}关键词:}{\songti\zihao{-4} XX;XX;XX}

\newpage    
% 论文标题(英文页)
\begin{center}
    {\zihao{3}\fontspec{Times New Roman} \textbf{The Development Status and Application Research of XXXX}}
\end{center}

\vspace{1.5cm}

% 摘要标题和内容
\noindent % 取消段落缩进
{\fontspec{Times New Roman}\zihao{4}\textbf{Abstract}}

\vspace{0.5\baselineskip} % 段前0.5行
\begin{spacing}{1.25} % 设置1.25倍行距
\fontspec{Times New Roman}\zihao{4} % 将字体和字号设置移到这里
Based on…
\end{spacing}

\vspace{0.5\baselineskip} % 段后0.5行

% 关键词
\noindent % 取消段落缩进
{\fontspec{Times New Roman}\zihao{4}\textbf{Key words:}}{\fontspec{Times New Roman}\zihao{-4} XX;XX;XX}

\newpage
% 引言
\section*{\centering \heiti\zihao{-2} \textbf{引\hspace{2em}言}}
\vspace{0.5cm}
\begin{spacing}{1.25} % 设置1.25倍行距
\songti\zihao{4} 
本文采用机器学习,构建了供应链延迟风险预测模型和延迟天数预测模型。通过对SCMS数据集10,324条历史交付记录的深入分析,识别出影响供应链绩效的关键风险因素,并依据风险评估结果制定了动态库存优化策略。
\end{spacing}

\newpage % 开始新的一页

% 目录
\begin{center}
    {\heiti\zihao{-2} 目\hspace{2em}录}\vspace* {-2\baselineskip}
\end{center}
\vspace{0.01cm}

% 设置目录格式
\renewcommand{\contentsname}{} % 清除默认的目录标题
\renewcommand{\cfttoctitlefont}{\hfill\heiti\zihao{3}} % 设置目录标题字体
\renewcommand{\cftaftertoctitle}{\hfill} % 设置目录标题后的格式
\renewcommand{\cftsecfont}{\songti\zihao{-4}} % 一级标题设置为宋体小四
\renewcommand{\cftsubsecfont}{\songti\zihao{5}} % 二级标题设置为宋体五号
\renewcommand{\cftsecpagefont}{\songti\zihao{-4}} % 一级标题页码字体
\renewcommand{\cftsubsecpagefont}{\songti\zihao{-4}} % 二级标题页码字体
\setlength{\cftbeforesecskip}{10pt} % 设置章节间距

% 在目录前强制设置页眉,并设置目录页页脚为空
\pagestyle{fancy}
\fancyhf{}
\renewcommand{\headrulewidth}{0pt}
\renewcommand{\footrulewidth}{0pt}
\chead{\color{gray}\songti\zihao{5}兰州资源环境职业技术大学毕业论文}
% \thispagestyle{empty} % 移动到tableofcontents之后

\setcounter{secnumdepth}{0} % 取消章节编号
\tableofcontents % 生成目录
\thispagestyle{empty} % 在生成目录后,但在clearpage前,设置当前页(目录页)为空页脚
\cleardoublepage % 清除当前页,确保新的一章从奇数页开始,并重置页码
\pagenumbering{arabic} % 从这里开始使用阿拉伯数字页码,并从1开始计数

% 在目录后再次强制设置页眉和页脚
\pagestyle{fancy}
\fancyhf{}
\renewcommand{\headrulewidth}{0pt}
\renewcommand{\footrulewidth}{0pt}
\chead{\color{gray}\songti\zihao{5}兰州资源环境职业技术大学毕业论文}
\cfoot{\color{gray}\songti\zihao{5}第~$\thepage$~页\quad 共~$\pageref{LastPage}$~页} % 设置页脚格式,页码使用数学模式,并添加空格
\thispagestyle{fancy} % 强制当前页为fancy

% \newpage % 由cleardoublepage处理分页
\section{\songti\zihao{3}{1\hspace{1em}绪论}}
\renewcommand{\thesection}{}
\begin{spacing}{1.25} % 设置1.25倍行距
\songti\zihao{4}
这里是绪论的引言部分,用于介绍研究的背景、目的和意义。\cite{HLKX202511017}

\subsection{\songti\zihao{4}1.1 背景介绍}
\renewcommand{\thesubsection}{}
\songti\zihao{4}
详细阐述研究的背景和重要性。

\subsection{\songti\zihao{4}1.2 公式的样例}
\renewcommand{\thesubsection}{}
\songti\zihao{4}
综述国内外相关领域的研究进展。

\subsection{\songti\zihao{4}1.3 文章目的与研究意义}
\renewcommand{\thesubsection}{}
\songti\zihao{4}
明确本研究的具体内容和采用的研究方法。

\subsection{\songti\zihao{4}1.4 论文结构安排}
\renewcommand{\thesubsection}{}
\songti\zihao{4}
简要介绍论文的章节安排。
\end{spacing}



\newpage
% 参考文献
\begin{spacing}{1.25} % 设置1.25倍行距
{\centering \heiti\zihao{-2} 参考文献\par}\vspace* {-2\baselineskip}
\songti\zihao{4}
\bibliographystyle{gbt7714-numerical} % 设置参考文献样式为国标数字样式
\bibliography{references} % 导入参考文献数据库文件 references.bib
\end{spacing}

\newpage
% 致谢
\begin{spacing}{1.25} % 设置1.25倍行距
{\centering \heiti\zihao{-2} 致谢\par}\vspace* {\baselineskip} % 致谢标题,黑体,小二号,居中,段后增加一些垂直间距
\songti\zihao{4} % 正文宋体四号
\noindent % 不缩进
在此处填写致谢内容。

例如:

感谢我的导师XXX教授的悉心指导。

感谢所有帮助过我的人。
\end{spacing}

\end{document}


